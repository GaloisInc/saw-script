\usepackage{listings}
\usepackage{float}
\usepackage{xspace}
\usepackage{color}
\usepackage{tikz}
\usepackage{url}
\usepackage{amsmath}
\usepackage{amscd}
\usepackage{verbatim}
\usepackage{fancyvrb}

\let\verbatiminput=\verbatimtabinput
\VerbatimFootnotes
\DefineVerbatimEnvironment{code}{Verbatim}{}
\DefineVerbatimEnvironment{pseudoCode}{Verbatim}{}
\hyphenation{SAW-Script}
\newcommand{\sawScript}{{\sc SAWScript}\xspace}
\renewcommand{\textfraction}{0.05}
\renewcommand{\topfraction}{0.95}
\renewcommand{\bottomfraction}{0.95}
\renewcommand{\floatpagefraction}{0.35}
\setcounter{totalnumber}{5}
\definecolor{MyGray}{rgb}{0.9,0.9,0.9}
\makeatletter\newenvironment{graybox}{%
   \begin{lrbox}{\@tempboxa}\begin{minipage}{\columnwidth}}{\end{minipage}\end{lrbox}%
   \colorbox{MyGray}{\usebox{\@tempboxa}}
}\makeatother

\setlength{\parskip}{0.6em}
\setlength{\abovecaptionskip}{0.5em}

\lstset{
         basicstyle=\footnotesize\ttfamily, % Standardschrift
         %numbers=left,               % Ort der Zeilennummern
         numberstyle=\tiny,          % Stil der Zeilennummern
         %stepnumber=2,               % Abstand zwischen den Zeilennummern
         numbersep=5pt,              % Abstand der Nummern zum Text
         tabsize=2,                  % Groesse von Tabs
         extendedchars=true,         %
         breaklines=true,            % Zeilen werden Umgebrochen
         keywordstyle=\color{red},
                frame=lrtb,         % left, right, top, bottom frames.
 %        keywordstyle=[1]\textbf,    % Stil der Keywords
 %        keywordstyle=[2]\textbf,    %
 %        keywordstyle=[3]\textbf,    %
 %        keywordstyle=[4]\textbf,   \sqrt{\sqrt{}} %
         stringstyle=\color{white}\ttfamily, % Farbe der String
         showspaces=false,           % Leerzeichen anzeigen ?
         showtabs=false,             % Tabs anzeigen ?
         xleftmargin=10pt, % was 17
         xrightmargin=5pt,
         framexleftmargin=5pt, % was 17
         framexrightmargin=-1pt, % was 5pt
         framexbottommargin=4pt,
         %backgroundcolor=\color{lightgray},
         showstringspaces=false      % Leerzeichen in Strings anzeigen ?
}

\author{The Galois SAW Team\\\texttt{saw@galois.com}}
\title{SAWScript Rust Tutorial}
