\documentclass[11pt]{amsart}
\usepackage[hmargin=.5in,vmargin=.5in]{geometry}
\geometry{letterpaper}
\usepackage{graphicx}
\usepackage[parfill]{parskip}
\usepackage{amssymb}
\usepackage{epstopdf}
\DeclareGraphicsRule{.tif}{png}{.png}{`convert #1 `dirname #1`/`basename #1 .tif`.png}


\renewcommand{\b}[1]{\textbf{#1}}

\title{SAWScript2: Overview}

\author{Brian Ledger}

\begin{document}
\maketitle

\section*{Abstract}

SAWScript is a language for the specification and dispatch of proofs involved in circuit verification and testing.  

\section*{Introduction}

\begin{itemize}
\item Refine the type system and syntax to express intuitive and clean data structures
\item Analyze workflows and propose optimizations in the syntax and standard library functions
\item Provide continuity between versions by identifying essential goals and idioms
\item Expose greater functionality with an improved standard library
\end{itemize}

\section*{Notes on the original SAWScript}

The original SAWScript was developed ad-hoc during the design verification of a suite of Java, LLVM/C, and Cryptol applications.  It aided in the orchestrated dispatch of proofs of equivalence between circuits, as well as the simulated testing of invariants.

One major achievement of SAWScript was to provide a framework for lifting stateful imperative methods in Java and C into stateless, well-typed, and functional specifications, suitable for formal methods verification.  This is achieved in part by modeling the contexts which affect and are affected by a method's execution, and then restating the method arguments and type to encompass these parameters.



\section*{SAWScript2: Core Concepts}

Note: Core Concepts are not necessarily final a.t.m., but represent ideas which have received the most interest and attention.

\begin{itemize}
\item \b{Bitfield Primitives}

Like Cryptol, SAWScript exposes arrays of bits and the functions between them as first class objects.  Distinguishing finite sets of bits allows the specification of properties that, in the context of streams or ``infinite" bitfields, would be formally undecidable.

\item \b{Atoms}

Unlike variables, atoms assume their syntactic names as an intrinsic property, allowing programmatic reflection and manipulation.  This is suitable for the concise specification of Java/LLVM contexts, wherein the name of a particular atom is indicative of it's role in the Java and LLVM simulators.

An atom declared without a binding or type-annotation is automatically assigned the ``don't care'' value (written explicitly as the underscore \b{\_}) and a bitfield type of \b{unknown}.  In other words,

\begin{verbatim}
myAtom;
\end{verbatim}

is a shorthand for the declaration

\begin{verbatim}
myAtom = _ : [?];
\end{verbatim}

\item \b{Bindings}

A binding is no more than the assignment to a variable, via the prefix

\begin{verbatim}
x = ...
\end{verbatim}

Any well-typed object can be assigned to a binding.

\item \b{Bags}

A bag is a heterogeneous collection of well-typed objects, separated by semi-colons, and delimited with braces.  Bags are a useful generic type, from which objects as various as contexts, proof-strategies, messages, and records can be derived.

\b{Example}

\begin{verbatim}

methodInputContext = {
  this.xBuf :: [8][8];
  this.W :: [80][64];
  this.H1; this.H2; this.H3; this.H4; this.H5; this.H6; this.H7; this.H8 }

> methodInputContext :: { [8][8]; [80][64]; [?]; [?]; [?]; [?]; [?]; [?]; [?]; [?] }

proofStrategy = { rewrite; yices }

> proofStrategy :: { rule }

\end{verbatim}

\item \b{Standard Libraries}
\begin{itemize}
\item makeContext
\item extractAIG, extractBTOR, extractJava, extractCryptol, extractLLVM, ...
\end{itemize}
\end{itemize}

\section*{SAWScript2: Proposed Concepts}

\begin{itemize}
\item Compositional Proofs (\b{JL} suggested)
\item Concurrent (Asynchronous?) Dispatch (\b{JL} suggested)
\item Rewrite Rules (complex, increased scope, needs more discussion)
\item Recursive Algebraic Datatypes
\item Enumerations
\end{itemize}

\section*{SAWScript2: Other Ideas}
\begin{itemize}
\item Decidable Checking of Bit-Width Arithmetic (bears investigation, potentially very rewarding)
\item Type Synonyms
\item Module System and Imports (goes well with Standard Libraries)
\end{itemize}

\section*{Functions and Function Composition}

\section*{}















\end{document}